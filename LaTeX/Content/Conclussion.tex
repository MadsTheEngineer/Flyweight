\section{Conclusion}
The flyweight pattern only makes sense 
if the objects share data, but it is most effective if there is a very large amount of objects or if the objects created are large.
The advantage is that instead of an expensive copy or move operation of an object a copy or move operation of a reference to the object can be used.
Several programming languages implement a flyweight solution for their handling of strings, called string interning.
String interning is, among others, implemented in Java, C\# and Python, but not in C or C++. Python has an implementation that checks the length of a string and only apply the pattern for short strings, under the philosophy that long strings are unlikely to be identical, which would mean that string interning wouldn't help at all. Because of this there could be situations, where the programmer would have to implement the pattern him/her self.
Since C and C++ does not have String interning implemented, it obviously makes sense to implement it in those languages as well.